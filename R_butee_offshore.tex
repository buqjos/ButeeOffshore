\documentclass{article}

% Gestion des fonts
\usepackage[utf8]{inputenc}
\usepackage[T1]{fontenc}

% Langue de r�daction
%\usepackage[british]{babel}

% Gestion des images
\usepackage{graphics}
\graphicspath{{./img/plot/},{./img/screenshot/}}
\DeclareGraphicsExtensions{.pdf,.png,.jpg}

% Gestion de l'esth�tique
\usepackage[textwidth=0.7\paperwidth, textheight=0.7\paperheight]{geometry}

% Commandes utilisateurs
\newcommand{\alinea}{\setlength{\parindent}{1cm}}
\renewcommand{\listfigurename}{Liste des figures}

% Page de garde
\title{Compte-rendu du BE But�e offshore}
\author{\bsc{Muller} Marc \and \bsc{Buquet} Joseph}

\begin{document}
Rapport
\section{Abstract}

\section{Introduction}

%TODO Unit�s utilis�es
% cm ^= 10 k Pa

%TODO Justification de la raideur transverse
% Explication de l'isogonflement lors de l'ajout de rondelles de m�tal

%TODO Choix de l'approximation lin�aire de la raideur
% Choix d'une erreur inf�rieure � 1% + au moins 5 points

%TODO Raideur initial
% Objectif 954 x 4 = 3816 N/cm

%TODO Justification choix lamelle
% Rayon identique � l'�lastom�re ou sup�rieur

\section{Approche du probl�me}

\subsection{Choix des param�tres de Mooney-Rivlin}
% Les notres : C10 0.2499 C01 0.0942 D1 0.0584
% Les siennes: C10 0.043  C01 0.495  D1 0.000658
% D�tails de la manip sur Abaqus + explication MR

\subsection{Choix du maillage}

\section{Conclusion}

\listoffigures

\begin{bibliography}{9}
% CONTACT_SEMINAR page 259
\end{bibliography}
\end{document}
\documentclass{article}

% Gestion des fonts
\usepackage[utf8]{inputenc}
%\usepackage[T1]{fontenc}

% Langue de rédaction
\usepackage[french]{babel}

% Gestion des images
\usepackage{graphics}
\graphicspath{{./Images/},{./Images/plot/}}

\DeclareGraphicsExtensions{.pdf,.png,.jpg}

% Gestion de l'esthètique
\usepackage[textwidth=0.7\paperwidth, textheight=0.7\paperheight]{geometry}

% Gestion des espaces
\usepackage{xspace}

% Commandes utilisateurs
\newcommand{\alinea}{\setlength{\parindent}{1cm}}
\newcommand{\abaqus}{\bsc{Abaqus}\xspace}
\renewcommand{\listfigurename}{Liste des figures}
% Bibliographie
% \bibitem[label 1]{cle 2} Auteur 3, TITRE 4, editeur 5, annee 6
\renewcommand{\bibitem}[6]{\bibitem[#1]{#2} \textsc{#3}, \emph{#4}, #5, #6}


% Page de garde
\title{Compte-rendu du BE Butée offshore}
\author{\bsc{Muller} Marc \and \bsc{Buquet} Joseph}

\begin{document}
\maketitle

\newpage

\section{Abstract}

\section{Introduction}

%TODO Justification de la raideur transverse
% Explication de l'isogonflement lors de l'ajout de rondelles de métal

%TODO Raideur initiale
% Objectif 954 x 4 = 3816 N/cm

%TODO Justification choix lamelle
% Rayon identique à l'élastomère ou supérieur

\section{Approche du problème}

%TODO Unités utilisées
\begin{tabular}{|l|c|}
\hline
Unités & Dimensions \\ \hline
Longueur & cm \\
Pression & k Pa \\
Force & N \\
Raideur & N\/cm \\ \hline
\end{tabular}

\subsection{Butée offshore}
% taille, forme
% compression
% materiau
La butée offshore étudiée est une butée cylindrique de rayon $R_0 = 70 cm$ et de hauteur $H_0 = 100 cm$ en Chloroprène X10. Dans le cadre de cette étude, la butée va être soumise à une compression afin d'évaluer sa raideur à une flèche de $\Delta H = 30 cm$.

Pour ce faire, une modélisation numérique de celle-ci va être faite sous \abaqus. Dans ce contexte, les hypothèses suivantes seront avancées :
\begin{itemize}
\item Matériau isotrope
\item Symétrie axiale
La géométrie axiale de la pièce est cylindrique. Le matériau est de plus isotrope. Enfin la contrainte de compression est homogène sur la surface supérieure de la pièce et agit parallèlement à l'axe de symétrie de la pièce.
\item Modélisation du matériau
Il est demandé de travailler avec une loi hyperélastique de type Mooney-Rivlin, les détails de cette approximation seront approfondis par la suite.
\end{itemize}

\subsection{Choix des paramètres de Mooney-Rivlin}

Il a été demandé pour ce travail d'utiliser une loi hyperélastique de type Mooney-Rivlin. Pour la mettre en place, deux résultats de test expérimentaux sont utilisés (cf. figure~\ref{fig:essais_exp} : 
\begin{itemize}
\item un test de traction uniaxiale
Ce test est réalisé sur une éprouvette de longueur nominale $25 mm$ et de section nominale $8 mm^2$ selon une plage de déplacement de $0$ à $104 mm$.
\item un test de traction plane
Ce test est réalisé sur une éprouvette de hauteur nominale $35 mm$ et de section nominale $392 mm^2$ selon une plage de déplacement de $0$ à $57.5 mm$.
\end{itemize}

\begin{figure}
\usegraphics{test_traction_uniaxiale}
\caption{Traction uniaxiale}
\usegraphics{test_traction_plane}
\caption{Traction plane}
\label{fig:essais_exp}
\end{figure}

Ces deux essais ont permit d'obtenir deux courbes de réponses en effort-déplacement (cf. figure~\ref{fig:plot_donnees_essais}). A partir de ces courbes, une approximation des coefficients de Mooney-Rivlin est lancée sur \abaqus. Les tables de points résultant des essais sont pour cela entrés dans \abaqus et une évaluation automatique des coefficients est lancée. Il en résulte les coefficients lus dans le tableau~\ref{tab:donnees_MR}. Ceux-ci sont bien évidemment à mettre en face de ceux données lors du début de ce projet comme référence pour mettre en exergue la forte influence de l'approximation faite dans ce cadre.

\begin{tabular}{|l|c|c|c|}
% Les notres : C10 0.2499 C01 0.0942 D1 0.0584
% Les siennes: C10 0.043  C01 0.495  D1 0.000658
\hline
 & C10 & C01 & D1 \\ \hline
Valeurs évaluées & 0.2499 & 0.0942 & 0.0584 \\ \hline
Valeurs données & 0.043  & 0.495 & 0.000658 \\ \hline
\label{tab:donnees_MR}
\end{tabular} 

Pour rester en cohérence avec les résultats obtenus à l'aide d'\abaqus, il sera choisit de conserver les valeurs dites évaluées comme coefficients de Mooney-Rivlin pour notre matériau. Nous justifions ce choix par la courbe~\ref{fig:plot_donnees_essais} qui synthétise les données expérimentales et les réponses force-déplacement obtenus avec les coefficients de Mooney-Rivlin.

\begin{figure}
\usegraphics{plot_donnees_essai}
\caption{Approximation du comportement du Chloroprène X10 par Mooney-Rivlin}
\label{fig:plot_donnees_essais}
\end{figure}

Afin d'étayer le choix de la méthode de Mooney-Rivlin dans cette étude, il est choisit de rappeler cette méthode.

\subsection{Explication de la méthode de Mooney-Rivlin}
%TODO écriture tensorielle
La méthode de Mooney-Rivlin est une méthode permettant de modéliser le potentiel élastique (ou de l'énergie de contrainte) d'un matériau. Pour cela, il est choisit une dépendance du potentiel élastique suivant les déviateurs des invariants $I_1$,$I_2$ du tenseur $B = V x V$ de Cauchy-Green gauche définit tel que $I_1 = Trace(B)$, $I_2 = \sec(B)$, $I_3=\det(\sigma)$.
On remarque que l'invariant $I_3$ n'est pas compris dans ce modèle. En effet, pour les élastomères, nous travaillons avec un effet Poisson $\nu = 0.49$, c'est-à-dire à volume quasi-constant. L'invariant $I_3$ est donc égal à 1 et n'est donc pas intéressant pour la modélisation de notre matériau puisqu'il n'évolue pas.

La méthode de Mooney-Rivlin résulte de l'amélioration du potentiel élastique de Mooney (1940) par Rivlin (1948) et statut la relation suivante \cite{wiki_MR}: $W = C_10(I_1-3) + C_01(I_2-3) + C_11(I_1-3)(I_2-3)$.
%TODO Corriger MR par l'ajout de barre sur les invariants.

De part l'expérience heuristique, il est assumé que cette approximation convient particulièrement bien dans la définition de l'hyperélasticité des elastomères spécialement sur des plages inférieure à 100\% de déformation en traction uniaxiale \cite{msc}.
%TODO avantages-inconvénients de MR
%Le modèle de Mooney-Rivlin peut être vu comme un cas particulier du modèle Ogden.

\subsection{Contact}
Lors de la compression de la butée offshore, l'élastomère se déforme à tel point que ses faces latérales viennent entrer en contact soit avec une lamelle d'acier ou avec lui-même. Dans ce cas, il faut définir une loi de contact pour éviter des problèmes d'interpénétration de la matière et obtenir des résultats convenables.

Dans cette étude, deux types de contact sont définis :
\begin{itemize}
\item le contact liant les parties aciers à l'élastomère
\item le contact pour les faces latérales de l'élastomère
\end{itemize}

Le premier contact est un contact de type \bsc{Tie} dans \abaqus, l'hypothèse est faite que l'interface entre la plaque d'acier et l'élastomère peut être considérée comme un encastrement. De manière classique, le maintien en position d'un élastomère et d'un acier est réalisé par une colle. Il est donc supposé que cette colle située à l'interface de nos deux pièces est rigides et qu'elle ne permet pas le mouvement des pièces ni en traction ni en cisaillement. Dans le cadre de cette étude où le comportement de notre élastomère est caractérisé en compression, il est raisonnable de croire que cette hypothèse n'influera qu'en très faible partie sur nos résultats et nous ne la considérerons plus par la suite.

Pour le deuxième type de contact à définir, c'est-à-dire entre les parties latérales de l'élastomères et soit les lamelles d'aciers soit leur symétrique par rapport aux lamelles, un contact de type \bsc{Penalty} est mis en place. Celui-ci sera détaillé par la suite.

\subsection{Calcul de la raideur}

La raideur caractérise la résistance à la déformation élastique d'un corps, c'est-à-dire la la valeur d'effort à lui appliquer pour engendrer une déformation donnée. Dans notre cas, le déplacement est imposé, et la simulation \abaqus nous fournit la force de réaction selon l'axe $\vect{y}$. La raideur $k$ de la butée offshore s'exprime ainsi:
\begin{displaymath}
		k=\frac{F_{y}}{y}
	\end{displaymath}
avec $y$ le déplacement axial imposé en compression et $F_{y}$ la force de réaction axiale associée à celui-ci. 

Dans l'ensemble de l'étude, la raideur $k$ sera caractérisée pour un déplacement de 30cm afin de pouvoir comparer celle-ci à la raideur des autres modèles proposés. Comme il n'y a pas de vibrations d'amplitude de déplacement autour d'une valeur quelconque dans ce cas, la raideur correspondra à la pente de la droite reliant l'origine et le point maximal de la courbe de force-déplacement générée par la simulation \abaqus, c'est-à-dire calculée comme suit:
\begin{displaymath}
		k=\frac{F_{y,max}}{30}
	\end{displaymath}
$k$ a pour unité XXXXXXXXXXXXX.


\subsection{Influence du maillage}

L'étude de la butée offshore a débuté par une étude de l'influence du maillage pour associer un maillage idéal à la pièce dans la modélisation \Abaqus. Ainsi, plusieurs simulations de compression ont été effectuées avec la butée initiale (voir section suivante, butée simple dans élements raidisseurs additionnels) pour différents paramètres de maille décroissants du plus grossier au plus fin: 4, 3, 2 et 1cm de largeur.
Une comparaison a consisté à comparer les courbes force-déplacement associés à chacun des maillages pour discerner leurs différences et ainsi leur influence sur les résultats de la simulation en compression. La figure~\ref{fig1} présente les réponses force-déplacement associées à plusieurs taille d'éléments S4R (de forme quadratique avec intégration réduite).

\begin{figure}[!h]
	\centering
	\includegraphics[width=\textwidth]{comparaison_mesh}
	\caption{Réponse force-déplacement de la butée pour différentes tailles de maille.}
	\label{fig1}
\end{figure}


On constate à première vue aucune différence entre les différentes courbes et par conséquent entre les différents maillages. Quand on zoom pour regarder précisément la valeur maximale de la force de réaction pour une flèche de 30cm, comme sur la figure~\ref{fig2}, on constate une lègère différence de 150XXX, c'est-à-dire de 0.8\%, entre un maillage de 4 et 2cm. Un pour un paramètre de maille plus fin, la valeur maximale de la force de réaction converge vers une même valeur.

\begin{figure}[!h]
	\centering
	\includegraphics[width=\textwidth]{comparaison_mesh_zoom}
	\caption{Réponse force-déplacement de la butée pour différentes tailles de maille - Zoom.}
	\label{fig2}
\end{figure}

La finesse du maillage n'a donc que très peu d'influence dans notre étude de la raideur de la butée. Un maillage de base d'élements de 2cm de largeur a été cependant choisi pour limiter le temps de calcul, bien qu'un maillage plus fin soit nécessaire dans beaucoup des modèles proposés où le contact avec les pièces additionneles est crutial.


\section{Butée initiale}

L'objectif de cette première partie de l'étude est de décrire le comportement en compression d'une butée offshore en élastomère libre de tout autre élément ou contrainte, afin de rendre compte de la raideur axiale de la pièce et ainsi pouvoir la comparer avec la raideur des autres modèles présentés dans la suite de l'étude.

La forme et le matériau de la butée ainsi que le déplacement imposé à ce dernier ont été décrits dans l'approche du problème. Les élements du maillage dont en S4R et ont pour paramètre 2mm. L'apparence du maillage initial de la pièce est présentée dans la figure~\ref{fig3}.

\begin{figure}[!h]
	\centering
	\includegraphics[width=\textwidth]{image_initial_mesh2}
	\caption{Visuel du maillage appliqué à la butée initiale.}
	\label{fig3}
\end{figure}

La figure~\ref{fig:fig4} permet de se rendre compte de l'écrasement subi par la butée. La déformation latérale maximale est de 23\%.

\begin{figure}[!h]
	\centering
	\includegraphics[width=\textwidth]{comparaison_mesh_zoom}
	\caption{Visuel de la butée initiale suite à la compression (déformations Ux affichées).}
	\label{fig4}
\end{figure}

La réponse force-déplacement de la butée initiale est affichée dans la figure~\ref{fig:fig5}

\begin{figure}[!h]
	\centering
	\includegraphics[width=\textwidth]{uf_initial_mesh2}
	\caption{Réponse force-déplacement de la butée initiale.}
	\label{fig5}
\end{figure}

On obtient donc une raideur de 631N/cm. Cette valeur servira de référence pour la suite de l'étude.


\section{Butée raidie par des lamelles}

<<<<<<< HEAD
Une première modification apportée à la butée initiale en vue de quadrupler sa raideur va consister à inister des lamelles horizontales réparties régulièrement sur la hauteur de la pièce avec deux diamètres différents: des lamelles circulaires du même diamètre que la butée en élastomère avant sa compression, que l'on nommera "lamelles imbriquées" et des lamelles avec un diamètre plus élevé que l'on nommera "lamelles étendues". On étudiera ce modèle renforcé d'une, deux et trois lamelles et on regardera l'influence du diamètre de ces dernières sur la raideur finale de la butée. Enfin, nous verrons que l'épaisseur de ces lamelles horizontales a peu d'influence sur les résultats.

\subsection{Lamelles imbriquées}

Un premier modèle d'insertion de lamelles horizontales circulaires de même rayon que l'élastomère à l'état initial (70cm) a été construit sur \abaqus. Il a consisté à diviser la \textit{part} "Butée" en 3 différentes \textit{part}, c'est-à-dire en insérant un élément au milieu des deux autres éléments constituants la butéee en élastomère.

L'épaisseur de la lamelle a été fixé arbitrairement\footnote{Enfin, -"arbitrairement"-, avec la prise en considération d'un certain ordre de grandeur relatif au matériau, tout de même.} à 0.5cm pour commencer les simulations. Cette valeur n'étant pas incohérente vis-à-vis des résultats dans la suite, elle a été conservée. Dans la partie XXXX, il sera montré que l'épaisseur a peu d'influence sur le résultat, et sachant que cette butée à une application industrielle, un plus faible dimensionnement pour une plus faible masse sera favorisé.

\subsection{Lamelles étendues}
%changement des paramètres de contact -> penalty avec k=10

\subsection{Influence de l'épaisseur des lamelles}
%àfaire



\section{Butée raidie par des anneaux}
%constraint coupling entre le point intérieur de la rondelle et le point exterieur de la butée
%self_contact aussi après


\section{Conclusion}

\listoffigures

\begin{thebibliography}{9}
%\bibitem{latexpratique} Christian \textsc{Rolland}. \emph{\LaTeX{} par la pratique}. O'Reilly, 1999.
%\bibitem[label]{cle} Auteur, TITRE, editeur, annee
%\bibitem{label 1}{cle 2}{Auteur 3}{TITRE 4}{editeur 5}{annee 6}
%\bibitem{msc}{mcstest}{MSC}{Nonlinear finite elements analysis of elastomers}{MSC Software}{20XX}
\bibitem{msc} Nonlinear finite elements analysis of elastomers
\bibitem{wiki_MR} https://en.wikipedia.org/wiki/Mooney%E2%80%93Rivlin_solid
% MSC Software, Nonlinear finite elements analysis of elastomers, MSC, 
% CONTACT_SEMINAR page 259
% Mooney Rivlin
%https://en.wikipedia.org/wiki/Mooney%E2%80%93Rivlin_solid
\end{thebibliography}
\end{document}
% TOREAD http://www.continuummechanics.org/mooneyrivlin.html
% TOREAD http://dmm.im.ufrj.br/~liu/Papers/MooneyRivlin.pdf

\documentclass{article}

% Gestion des fonts
\usepackage[utf8]{inputenc}
\usepackage[T1]{fontenc}

% Langue de r�daction
\usepackage[french]{babel}

% Gestion des images
\usepackage{graphics}
\graphicspath{{./img/plot/},{./img/screenshot/}}
\DeclareGraphicsExtensions{.pdf,.png,.jpg}

% Gestion de l'esth�tique
\usepackage[textwidth=0.7\paperwidth, textheight=0.7\paperheight]{geometry}

% Commandes utilisateurs
\newcommand{\alinea}{\setlength{\parindent}{1cm}}
\renewcommand{\listfigurename}{Liste des figures}
% Bibliographie
% \bibitem[label 1]{cle 2} Auteur 3, TITRE 4, editeur 5, annee 6
\renewcommand{\bibitem}[6]{\bibitem[#1]{#2} \textsc{#3}, \emph{#4}, #5, #6}

% Page de garde
\title{Compte-rendu du BE But�e offshore}
\author{\bsc{Muller} Marc \and \bsc{Buquet} Joseph}

\begin{document}
Rapport
\section{Abstract}

\section{Introduction}

%TODO Unit�s utilis�es
% cm ^= 10 k Pa

%TODO Justification de la raideur transverse
% Explication de l'isogonflement lors de l'ajout de rondelles de m�tal

%TODO Choix de l'approximation lin�aire de la raideur
% Choix d'une erreur inf�rieure � 1% + au moins 5 points

%TODO Raideur initial
% Objectif 954 x 4 = 3816 N/cm

%TODO Justification choix lamelle
% Rayon identique � l'�lastom�re ou sup�rieur

\section{Approche du probl�me}

\subsection{Choix des param�tres de Mooney-Rivlin}
% Les notres : C10 0.2499 C01 0.0942 D1 0.0584
% Les siennes: C10 0.043  C01 0.495  D1 0.000658
% D�tails de la manip sur Abaqus + explication MR

%TODO D�finition du potentiel �lastique...
Description de la m�thode de Mooney-Rivlin

La m�thode de Mooney-Rivlin est une m�thode permettant de mod�liser le potentiel de l'�nergie de contrainte d'un mat�riau. Pour cela, il est �tabli une relation entre le travail du mat�riau et les invariants $I_1$,$I_2$,$I_3$ d�finit tel que $I_1 = Trace(\sigma)$, $I_2 = \sec(\sigma)$, $I_3=\det(\sigma)$.
La m�thode de Mooney-Rivlin d�finit en 1940(Mooney) statue la relation suivante : $W = C_10(I_1-3) + C_01(I_2-3) + C_11(I_1-3)(I_2-3)$.

Le mod�le de Mooney-Rivlin peut �tre vu comme un cas particulier du mod�le Ogden.

On remarque que l'invariant $I_3$ n'est pas compris dans ce mod�le. En effet, pour les �lastom�res, nous travaillons avec un effet Poisson $\nu = 0.49$, c'est-�-dire � volume quasi-constant. L'invariant $I_3$ est donc �gal � 1 et n'est donc pas int�ressant pour la mod�lisation de notre mat�riau.

\subsection{Contact}
Lors de la compression de la but�e offshore, il peut arriver que deux parties de l'�lastom�re rentrent en contact. Dans ce cas, il faut d�finir des lois de contacts pour les g�rer. 

\subsection{Choix du maillage}

\section{Conclusion}

\listoffigures

\begin{bibliography}{9}
%\bibitem{latexpratique} Christian \textsc{Rolland}. \emph{\LaTeX{} par la pratique}. O'Reilly, 1999.
%\bibitem[label]{cle} Auteur, TITRE, editeur, annee
%\bibitem{label 1}{cle 2}{Auteur 3}{TITRE 4}{editeur 5}{annee 6}
\bibitem{msc}{mcstest}{MSC}{Nonlinear finite elements analysis of elastomers}{MSC Software}{20XX}

% MSC Software, Nonlinear finite elements analysis of elastomers, MSC, 
% CONTACT_SEMINAR page 259
% Mooney Rivlin
%https://en.wikipedia.org/wiki/Mooney%E2%80%93Rivlin_solid
\end{bibliography}
\end{document}
% TOREAD http://www.continuummechanics.org/mooneyrivlin.html
% TOREAD http://dmm.im.ufrj.br/~liu/Papers/MooneyRivlin.pdf
